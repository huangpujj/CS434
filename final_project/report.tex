\documentclass[11pt,a4paper]{article}
\usepackage{amsmath}
\usepackage{graphicx}
\usepackage{geometry}
 \geometry{
 a4paper,
 total={170mm,257mm},
 left=20mm,
 top=20mm,
 }
%\usepackage[colorinlistoftodos]{todonotes}

\title{CS434 Final Project Report (3-page limit)}
\author{Kin-Ho Lam, Donald Elkins, Kien Tran}
\date{}
\begin{document}
\maketitle
\section{Feature formulation and preprocessing}
\subsection{Features} What are the features you feed to your learning algorithm? Did you simply flatten the 7 rows into a vector for features? Did you transform or aggregate the given data to engineer your own features?

Given a window size of 7 events, where each event contains a value for glucose, slope, IOB, MOB, morning, afternoon, and night, we flattened the features into a 56-long vector.
The last event's hypo value is used as the class label for that set of data.
In the CNN model for subject 2, I converted the flattened data into a 7x8 matrix.

\subsection{Preprocessing}
Did you pre-process your data in any way? This can be for the purpose of reducing dimension, or reducing noise, or balancing the class distribution. Be clear about what you exactly did. The criterion is to allow others to replicate your works.

We did not pre-process the data.
Given the poor results from our NN and CNN models, one can presume that some pre-processing is necessary to add more distinction between classes or remove noise.

\section{Learning algorithms}
\subsection{Algorithms explored}
Provide a list of learning algorithms that you explored for this project. For each algorithm, briefly justify your rationale for choosing this algorithm.

\begin{enumerate}
	\item 1 and 2 layer Neural Network
		\subitem
		I created a neural network model using pytorch for a general and subject 1 model.
		I chose to experiment with a neural network model for general and subject 1 because I have had past success with creating models for non-linearly separable data.
		It is clear from my results these models over-fit the training data, as both had very high AUC and precision values but does not produce realistic results in the final test data.

	\item 1 layer Convolution Neural Network
		\subitem
		I created a CNN model using pytorch for subject 2.
		I chose to experiment with a CNN because I wasn't producing good results from a NN model.
		It is clear from my results that I am either doing something wrong, my data format is not appropriate for a CNN, or models over-fit the training data, as both had very high AUC and precision values but does not produce realistic results in the final test data.
		
	\item Gaussian Naive Bayes
	    \subitem
	    I used Gaussian Naive Bayes because I saw each of the variables described in the assignment as being independent of one another.
	    Due to their independence assumption, we do not need to make any correlation between each feature into consideration and can use what we have to test for the hypoglycemic event.
	    Another reason is how fast naive Bayes takes to train and run.
	    Due to their low computational complexity, using a Bayes to train multiple models and adjusting for variations is simpler.
	    Looking back now, realizing that we are using a flatten vector array of 56 features, it would be better to assume that some variables do relate to one another and try to using a different classifier based off that.
	    
	 \item Batch Gradient Descent Linear Classifier
	    \subitem
	    The relatively low number of features per instance in this dataset made a linear classifier via gradient-descent appear to be a feasible learning algorithm. The math is (supposedly) simple and straightforward and the weights generated from the training phase can be saved and applied to other datasets as a model. I did, unfortunately, run into significant issues with normalization and rounding which ended up undermining the model's effectiveness.
\end{enumerate}


\subsection{Final models}
What are the final models that produced your submitted test predictions?
\begin{enumerate}
	\item 1 and 2 layer Neural Network
	\item Single layer Convolution Neural Network
	\item Gaussian Naive Bayes
	\item Batch Gradient Descent Linear Classifier
\end{enumerate}

\section{Parameter Tuning and Model Selection }
\subsection{Parameter Tuning}
What parameters did you tune for your models? How do you perform the parameter tuning?
\begin{enumerate}
	\item NN and CNN
		\subitem
		I tuned the learning rate, epoch, batch size, and number of neurons in each layer/hidden layer/convoluted layer.
		I also adjusted the kernel size, loss function, and optimizer function.
		
	\item Gaussian Naive Bayes
	    \subitem
	    I was able to tune the weight of the Bayesian network as well as what standardization to use when standardizing the data.
	    The weights would play a key roll in controlling the response variable and standardizing the weights helps keep one variable from overpowering the others.
	    
	 \item Batch Gradient Descent
	   \subitem
	   I performed hand tuning of the learning rates, number of iterations (epochs), and the k-fold values to try to get optimum learning rates and accuracy.
\end{enumerate}


\subsection{Model selection}
How did you decide which models to use to produce the final predictions?  Do you use cross-validation or hold-out for model selection? When you split the data for validation, is it fully random or special consideration went into forming the folds? What criterion is used to select the models?

\begin{enumerate}
	\item NN and CNN
		\subitem
		I cherry picked the best performing models based on the results of the testing script.
		Due to the nature of k-folds, the training and testing assignments are fully random.
	\item Gaussian Naive Bayes
	    \subitem
	    After finding out which standardization to use, it was simple to keep the model I have since it perform the best through different test runs.
	    I ran different models and recorded their best recall, precision, F1 and AUC scores from the scripted provided by the TAs.
	    In order to do this, I have to use KFolds and hold back some data for validation.
	 \item Batch Gradient Descent
	    \subitem
	    The choice essentially came down to semi decent test runs that I was getting with my algorithm. I was unable to troubleshoot my sigmoid normalization which kneecapped the program. There was k-folds and cross-validation, so I'd like to think that the classifier (if the sigmoid and normalization were working properly and there weren't weird arithmetic errors I couldn't catch) is quite robust.

\end{enumerate}

\section{Results}
Do you have any internal evaluation results you want to report?


Our results do not seem realistic.
It seems that some data pre-processing is needed to prepare the dataset before it is fed into a NN or CNN.

The result from the Gaussian Naive Bayes with a kFolds = 20:
\begin{center}
    \begin{tabular}{||c || c c c c||}
        \hline
        Data & Precision & Recall & F1 & AUC \\ [0.5ex] 
        \hline\hline
        Group & 0.001 & 1.0 & 0.002 & 0.23 \\ 
        \hline
        Individual\_2 & 0.018 & 1.0 & 0.035 & 0.161 \\
        \hline
        Individual\_7 & 0.018 & 1.0 & 0.035 & 0.435 \\
        \hline
    \end{tabular}\par
    \bigskip
    Results of Gaussian Naive Bayes
\end{center}
As you can see from the chart, our Naive Bayes classifier did not preform well.
However, an interesting note is that the AUC is higher for individual\_7 than the group data.
However, the values is low that is not even better than a random guess.

The results from the batch gradient descent algorithm were also poor, as referenced above. I was able to reach relatively high precision and AUC, but recall and F1 were consistently 0, which indicates to me that the algorithm failed. This is consistent with what I was seeing from output files where the script was unable to classify anything as 1 because of the rounding and normalization errors. The accuracy should still be relatively high, but this is only because at the current moment the script effectively classifies everything as negative (and thus adheres to the general statistical trend of the data).

\end{document}